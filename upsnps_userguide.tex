\documentclass{article}
\usepackage[a4paper,margin=1in]{geometry}
\usepackage{enumitem}

\title{UPS vs NPS Calculator - User Guide}
\author{Yogesh Wadadekar}
\date{\today}

\begin{document}
\maketitle

\section{Introduction}
This calculator helps compare the University Pension Scheme (UPS) with the National Pension Scheme (NPS) by estimating retirement benefits under both schemes. It takes into account current salary, growth rates, and existing NPS corpus to provide a comprehensive analysis.

\section{Running the Calculator}
You can run the calculator in two ways:

\subsection{Browser Interface (Recommended)}
\begin{enumerate}
    \item Open the index.html file in a web browser
    \item Fill in the values in the form or use the default values
    \item Click the "Calculate Results" button to see the analysis
\end{enumerate}

The browser interface provides a user-friendly form with:
\begin{itemize}
    \item Clear sections for different types of inputs
    \item Pre-filled default values
    \item Instant results without needing to install Python
    \item Formatted output that's easy to read
\end{itemize}

\subsection{Command Line Interface}
Alternatively, you can use the Python command line interface:
\begin{enumerate}
    \item Open a terminal
    \item Navigate to the directory containing upsnpscalculator.py
    \item Run: python upsnpscalculator.py
\end{enumerate}

\section{Input Parameters}
The calculator will prompt for various inputs. You can press Enter to use the default values shown in brackets. Here are the parameters with their default values and explanations:

\subsection{Basic Information}
\begin{itemize}
    \item Current Age [53]
    \item Retirement Age [60]
    \item Age when you joined Government service [28] \\
          Used to calculate total years of qualifying service.
    \item Current Annual (Basic + DA) in Lakhs [36.00] \\
          This is your current Basic Pay plus Dearness Allowance
    \item Annual Salary Growth Rate [7\%] \\
          Historical average growth rate in government sector
    \item Current NPS Corpus in Lakhs [120.00] \\
          Your accumulated NPS amount so far
    \item Expected Years of Life After Retirement [20] \\
          Number of years employee is expected to live after retirement
    \item Additional Years Spouse May Live [10] \\
          Number of additional years spouse may live after employee's death
\end{itemize}

\subsection{NPS Parameters}
\begin{itemize}
    \item Employee Contribution Rate [10\%] \\
          Standard employee contribution rate as per government rules
    \item Employer Contribution Rate [14\%] \\
          Current government contribution rate to NPS
    \item Expected Annual Return on NPS [9.5\%] \\
          Conservative estimate based on historical NPS returns
    \item Annuity Conversion Rate without return of purchase price [7\%] \\
          Default annuity rate used for pension conversion without return of purchase price
\end{itemize}

\subsection{Post-Retirement Parameters}
\begin{itemize}
    \item Post-retirement UPS Pension Growth [5\%] \\
          Based on historical DA increase patterns
    \item Return on Remaining NPS Corpus [8\%] \\
          Expected return on the 60\% lump sum amount if invested
\end{itemize}

\section{Output Explanation}
The calculator provides:
\begin{itemize}
    \item Final basic salary at retirement
    \item Monthly pension under UPS scheme for:
    \begin{itemize}
        \item Employee (Proportional to service, up to 50\% of final basic salary if service is 25 years or more)
        \item Spouse (60\% of employee's pension after employee's death)
    \end{itemize}
    \item UPS lump sum amount at retirement, calculated as 1/10th of the last drawn monthly basic pay (plus Dearness Allowance) for every completed six months of qualifying service.
    \item Projected value of the UPS lump sum if invested at the same post-tax rate as the NPS 60\% lump sum.
    \item Yearly UPS return on investment (added to UPS pension for comparison)
    \item Total NPS corpus at retirement
    \item Monthly pension from NPS annuity (40\% of corpus)
    \item Lump sum amount available (60\% of corpus)
    \item Year-by-year analysis showing:
    \begin{itemize}
        \item Separate phases for employee and spouse periods
        \item UPS pension amount (reducing to 60\% for spouse)
        \item UPS lump sum return (added to UPS pension for comparison)
        \item NPS annuity amount (remains constant for both phases)
        \item \textbf{Yearly difference (UPS-NPS):} This is calculated as the difference between total UPS income (UPS pension + UPS lump sum return) and total NPS income (NPS annuity + NPS corpus return) for each year.
        \item \textbf{Shortfall subtraction timing:} The shortfall (if any) is only subtracted from the NPS corpus at the end of each year, after the corpus has grown by the investment return for that year. This means that even if the NPS return is less than the UPS return, the corpus can increase in the initial years if the investment return is high enough, and the shortfall is not immediately subtracted each month but only once at year-end.
        \item Interest earned on remaining corpus (NPS 60\% lump sum)
        \item Interest earned on UPS lump sum (if invested)
        \item Remaining corpus balance
    \end{itemize}
    \item Analysis of whether the corpus will last through both employee and spouse lifetimes. The corpus is considered perpetual if NPS annuity + NPS corpus return matches or exceeds UPS pension + UPS lump sum return.
\end{itemize}

\section{Important Notes}
\begin{itemize}
    \item All monetary inputs (salary and corpus) are in lakhs for convenience
    \item Rates should be entered as decimals (e.g., 0.07 for 7\%)
    \item UPS pension calculation:
    \begin{itemize}
        \item Employee receives a pension proportional to their years of service. If service is 25 years or more, pension is 50\% of final basic salary. Otherwise, it is (years of service / 25) * 50\% of final basic salary.
        \item Spouse receives 60\% of employee's pension after employee's death
        \item UPS pension grows at specified rate (default 5\%) for both phases
    \end{itemize}
    \item NPS features:
    \begin{itemize}
        \item Only 40\% of NPS corpus is mandatory for annuity purchase
        \item Annuity amount remains constant for both employee and spouse
        \item The remaining 60\% lump sum can be invested to cover UPS-NPS difference
        \item The calculator shows if this corpus will last through both lifetimes
    \end{itemize}
\end{itemize}

\section{Browser Interface Features}
The browser interface includes several convenient features:
\begin{itemize}
    \item Input validation to prevent errors
    \item Clearly organized sections for:
    \begin{itemize}
        \item Basic Information
        \item NPS Contribution Rates
        \item Return Rates
        \item Life Expectancy
    \end{itemize}
    \item No need to press Enter for defaults - values are pre-filled
    \item Immediate recalculation with new inputs
    \item Results displayed in a clean, formatted layout
    \item Ability to easily try different scenarios by changing values
    \item \textbf{Reset to Defaults} button to quickly restore all input fields to their original default values
\end{itemize}

\end{document}